%-*- coding: iso-latin-1 -*-
\documentclass[french,11pt]{article}
\usepackage{babel}
\DecimalMathComma
% Emacs: to save in encoding iso-latin-1:
% C-x C-m f
% iso-latin-1

% aspell --lang=fr --encoding='iso-8859-1' -t check selection-modele.tex

\usepackage{graphicx}
\usepackage[hidelinks]{hyperref}


% Fonts
\usepackage[latin1]{inputenc}
\usepackage[T1]{fontenc}
\usepackage{gentium}

% SI units
\usepackage{siunitx}

% Table becomes Tableau
\usepackage{caption}
\captionsetup{labelfont=sc}
\def\frenchtablename{Tableau}

% % List management
\usepackage{enumitem}

\usepackage[dvipsnames]{xcolor}
\usepackage{listings}
\lstset{%
  frame=single,                    % adds a frame around the code
  tabsize=2,                       % sets default tabsize to 2 spaces
  columns=flexible,                % doesn't add spaces to make the line fit the whole column
  basicstyle=\ttfamily,             % use monospace
  keywordstyle=\color{MidnightBlue},
  commentstyle=\color{Gray},
  stringstyle=\color{BurntOrange},
  showstringspaces=false,
}


%%%% GEOMETRY AND SPACING %%%%%%%%%%%%%%%%%%%%%%%%%%%%%%%%%%%%%%%%%%%%%%%
\usepackage{etex}
\usepackage[tmargin=2cm,bmargin=2cm,lmargin=2cm,footnotesep=1cm]{geometry}

\parskip=1ex\relax % space between paragraphs (incl. blank lines)
%%%%%%%%%%%%%%%%%%%%%%%%%%%%%%%%%%%%%%%%%%%%%%%%%%%%%%%%%%%%%%%%%%%%%%

\input{contents/notations}

\begin{document}

\begin{center}
\bf\large ECUE21.2: Science des donn�es \hfill
Erratum
\end{center}
\noindent
\rule{\textwidth}{.4pt}

\medskip

\section{Fonction de co�t logistique et entropie crois�e (Section 7.4.2, p. 75)}

� propos de la phrase � Si $f$ est � valeurs dans $]0, 1[$, en particulier si $f(\xx)$ est la
probabilit� que $\xx$ appartienne � la classe positive, cette fonction de co�t est �quivalente �
l'\textbf{entropie crois�e}, d�finie pour $\YY = \zo$ � : cette �quivalence est vraie si 
l'on transforme $f,$ � valeurs dans $\RR,$ en une fonction $h$ � valeurs dans $]0, 1[$ en la composant par la
\textbf{fonction sigmo�de,} aussi appel�e \textbf{fonction logistique,} d�finie
par
\begin{equation}
  \label{eq:sigmoide}
  \begin{split}
    \sigma \colon \RR & \to ]0, 1[ \\
    u & \mapsto \frac1{1 + e^{-u}}.
  \end{split}
\end{equation}
Dans ce cas, la fonction de co�t logistique appliqu�e � $f$ est �quivalente �
l'entropie crois�e appliqu�e � $h = \sigma \circ g$.


\end{document}

%%% Local Variables:
%%% mode: latex
%%% TeX-master: t
%%% End:
