%-*- coding: iso-latin-1 -*-
\section{QCM}
\paragraph{Question 1.} Un mod�le de r�gression r�gularis�e est plus
  susceptible de surapprendre si le param�tre de r�gularisation est
\begin{itemize}
\item[$\square$] �lev� ;
\item[$\square$] faible ;
\item[$\square$] �a d�pend des cas.
\end{itemize}

\paragraph{Question 2.}     Dans un lasso, il y a plus de coefficients nul quand le 
    param�tre de r�gularisation est 
\begin{itemize}
\item[$\square$] �lev� ;
\item[$\square$] faible ;
\item[$\square$] �a d�pend des cas.
\end{itemize}

\paragraph{Question 3.} Par rapport � un mod�le complexe, un mod�le plus simple est
\begin{itemize}
\item[$\square$] plus rapide � entra�ner ;
\item[$\square$] plus susceptible de surapprendre ;
\item[$\square$] plus susceptible de bien g�n�raliser ;
\item[$\square$] plus susceptible de minimiser le risque empirique.
\end{itemize}


\section*{Solution}
{%
\noindent
\rotatebox[origin=c]{180}{%
\noindent
\begin{minipage}[t]{\linewidth}
\paragraph{Question 1.}  Quand $\lambda$ est faible, c'est le risque empirique
qui domine et le mod�le est plus susceptible de surapprendre. \newline

\paragraph{Question 2.} Quand $\lambda$ cro�t, le r�gulariseur prend plus
d'importance et le nombre de coefficients nuls augmente. \newline

\paragraph{Question 3.} Le temps d'entrainement ne d�pend pas toujours de la
complexit� du mod�le. Un mod�le plus simple sera cependant souvent plus
rapide � entrainer.

Un mod�le simple est moins susceptible de surapprendre (et plus susceptible de sousapprendre) ; g�n�ralisera mieux, sauf s'il est en sous-apprentissage ; et minimisera moins bien le risque empirique.
\end{minipage}%
}%







