\section{QCM}
\paragraph{Question 1.} On s'intéresse aux hospitalisations pour une certaine
maladie. Comment visualiser la liaison entre la durée du séjour à l'hôpital et
l'âge des patients, la première étant donnée en nombre de jours et le second
par tranches ?
\begin{itemize}
\item[$\square$] un nuage de points 
\item[$\square$] un diagramme en barres
\item[$\square$] une série de boîtes à moustaches
\end{itemize}

\paragraph{Question 2.} L'image ci-dessous représente un nuage de points entre
le diamètre de fleurs et la hauteur de leur tige. Leur coefficient de
corrélation de Pearson est plutôt proche de...
\begin{itemize}
\item[$\square$] $- 0,35$
\item[$\square$] $+ 0,35$
\item[$\square$] $- 0,85$
\item[$\square$] $+ 0,85$
\item[$\square$] $- 0,95$
\item[$\square$] $+ 0,95$
\item[$\square$] $- 0,50$
\item[$\square$] $+ 0,50$
\end{itemize}

\vspace{-13em}
\begin{center}
  \includegraphics[width=0.35\textwidth]{figures/pearson_example}
\end{center}



\section*{Solution}
{%
\noindent
\rotatebox[origin=c]{180}{%
\noindent
\begin{minipage}[t]{\linewidth}
  \paragraph{Question 1.} Une série de boîtes à moustaches est plus appropriée
  pour visualiser la relation entre une variable quantitative (durée du séjour)
  et une variable qualitative (âge par
  tranches). Cf. figure~\ref{fig:remboursement_rembourses_age}.\newline

\paragraph{Question 2.} $r \approx 0,85.$ On peut voir à la « pente » que la
corrélation est positive. La situation est intermédiaire entre celle des
figures 2.6(\textsc{C}) $(r=0,50)$ et 2.6(\textsc{D}) $(r=1,00)$. Une
corrélation de $0,95$ serait plus proche de la figure~\ref{fig:pearson}(\textsc{D}) que de
celle donnée ci-dessus. Remarquez ici que les données ne sont pas homogènes, au
sens où elles ont des échelles de valeurs différentes, contrairement à ce qui
est représenté sur la figure~\ref{fig:pearson} ; cela ne change pas l'interprétation de la
corrélation.
\end{minipage}%
}%


%%% Local Variables:
%%% mode: latex
%%% TeX-master: "../../sdd_2021_poly"
%%% End:
